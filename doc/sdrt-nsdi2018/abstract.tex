\subsection*{Abstract}
Increasing pressure for higher throughput, better response times, and lower cost
in datacenters have pushed researches to explore alternative technologies for
network fabrics (e.g., 60GHz wireless, free-space optics, optical circuit
switching). While these alternatives can theoretically meet the demand, they all
share a common limitation: some links must be pulled down for other links to be
up. This limitation can be overcome by specialized software schedulers, thus we
dub these technologies ``Software Defined Reconfigurable Topology'' (SDRT)
networks. While specific SDRT technologies differ, we find little work on
apples-to-apples comparisons nor end-to-end optimization. Thus, in this work we
make two primary contributions: 1) an open-source framework for evaluating SDRTs
end-to-end on public testbeds, and 2) an analysis of the three key
``choke-points'' (the switch, the network stack, and applications) where
cross-layer optimizations improve end-to-end performance. We implement one
mechanism in each choke-point: adaptive ToR switch buffer resizing based on the
SDRT schedule, a socket API interposition library that provides ADU size
information to the SDRT scheduler, and a modified HDFS replica placement
strategy that provides easier to schedule demand. All of these cross-layer
optimizations fundamentally focus on improving the network schedule: adaptive
buffer resizing improves TCP performance on the running schedule, communicating
ADU sizes allows data to be scheduled before it reaches the ToR switch, and
changing replica placement changes what traffic matrix needs to be scheduled.
We additionally explore the impact of these optimizations on four real-world
datacenter applications.

%% End-to-end improvements in SDRTs are fundamentally network scheduling
%% improvements; either the host make more efficient use of their scheduled time,
%% the scheduler produces schedules more aligned with traffic, or more
%% schedule-friendly traffic is produced.

%% Our analysis shows these three key points have different pain/benefit tradeoffs:
%% 1) modifications to the SDRT ``switch'' are easy to deploy but provide only
%% small benefits, 2) modifications to the networking stack are harder but provide
%% more benefits, and 3) modifications to applications, while hard to deploy
%% provide the most benefit.


%% Recently there have been many exciting technologies proposed for use as
%% datacenter fabrics (e.g., 60GHz wireless, free-space optics, optical circuit
%% switches), each with their own unique technical constraints. These technologies
%% all share a common limitation: which nodes can communicate at a given time is
%% constrained; changing which nodes can communicate requires (automated) topology
%% reconfiguration. Proper reconfiguration of the topology is key to provide
%% reasonable performance. Prior work treat each of these new network technologies
%% as point solutions, with little focus on how each relate to each other in the
%% broader datacenter context. The goal of this line of work is to provide an
%% generalized framework that each of these new technologies naturally fits into
%% that can optimize network usage based on the technology's specific needs. This
%% paper takes the first steps towards this goal by characterizing the space in
%% which these technologies reside, as well as looking at how network-level
%% primitives (e.g., multicast, anycast) in addition to cross-layer optimization
%% can greatly improve application completion time over these technologies. Our
%% initial simulation results show that for certain workloads we can achieve an
%% xx\% decrease in application completion time over a naive baseline without
%% modifying applications and a xxx\% decrease in completion time with application
%% modification.


%% A range of new datacenter switch designs combine wireless or optical circuit
%% technologies with electrical packet switching to deliver higher performance at
%% lower cost than traditional packet-switched networks.  These ``hybrid'' networks
%% schedule large traffic demands via a high-rate circuits and remaining traffic
%% with a lower-rate, traditional packet-switches. Achieving high utilization
%% requires an efficient scheduling algorithm that can compute proper circuit
%% configurations and balance traffic across the switches. Recent proposals,
%% however, provide no such algorithm and rely on an omniscient oracle to compute
%% optimal switch configurations.

%% Finding the right balance of circuit and packet switch use is difficult:
%% circuits must be reconfigured to serve different demands, incurring non-trivial
%% switching delay, while the packet switch is bandwidth constrained. Adapting
%% existing crossbar scheduling algorithms proves challenging with these
%% constraints. In this paper, we formalize the hybrid switching problem, explore
%% the design space of scheduling algorithms, and provide insight on using such
%% algorithms in practice. We propose a heuristic-based algorithm, Solstice
%% %% , that takes advantage of skew and sparsity in typical datacenter traffic
%% %% demand matrices to
%% that provides a 2.9$\times$ increase in circuit utilization over traditional
%% scheduling algorithms, while being within 14\% of optimal, at scale.
