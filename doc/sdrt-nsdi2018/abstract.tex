\subsection*{Abstract}
Increasing pressure for higher throughput, better response times, and lower cost
in datacenters have pushed researches to explore alternative technologies for
network fabrics (e.g., 60GHz wireless, free-space optics, optical circuit
switching). While these alternatives can theoretically meet the demand, they all
share a common limitation: which nodes can communicate at a given time are
limited due to long switching times. These constraints ``cut'' links for brief
periods, reconfiguring the topology. 

Applications in datacenters continuously push the boundaries of what the network
can provide



Recently there have been many exciting technologies proposed for use as
datacenter fabrics (e.g., 60GHz wireless, free-space optics, optical circuit
switches), each with their own unique technical constraints. These technologies
all share a common limitation: which nodes can communicate at a given time is
constrained; changing which nodes can communicate requires (automated) topology
reconfiguration. Proper reconfiguration of the topology is key to provide
reasonable performance. Prior work treat each of these new network technologies
as point solutions, with little focus on how each relate to each other in the
broader datacenter context. The goal of this line of work is to provide an
generalized framework that each of these new technologies naturally fits into
that can optimize network usage based on the technology's specific needs. This
paper takes the first steps towards this goal by characterizing the space in
which these technologies reside, as well as looking at how network-level
primitives (e.g., multicast, anycast) in addition to cross-layer optimization
can greatly improve application completion time over these technologies. Our
initial simulation results show that for certain workloads we can achieve an
xx\% decrease in application completion time over a naive baseline without
modifying applications and a xxx\% decrease in completion time with application
modification.


%% A range of new datacenter switch designs combine wireless or optical circuit
%% technologies with electrical packet switching to deliver higher performance at
%% lower cost than traditional packet-switched networks.  These ``hybrid'' networks
%% schedule large traffic demands via a high-rate circuits and remaining traffic
%% with a lower-rate, traditional packet-switches. Achieving high utilization
%% requires an efficient scheduling algorithm that can compute proper circuit
%% configurations and balance traffic across the switches. Recent proposals,
%% however, provide no such algorithm and rely on an omniscient oracle to compute
%% optimal switch configurations.

%% Finding the right balance of circuit and packet switch use is difficult:
%% circuits must be reconfigured to serve different demands, incurring non-trivial
%% switching delay, while the packet switch is bandwidth constrained. Adapting
%% existing crossbar scheduling algorithms proves challenging with these
%% constraints. In this paper, we formalize the hybrid switching problem, explore
%% the design space of scheduling algorithms, and provide insight on using such
%% algorithms in practice. We propose a heuristic-based algorithm, Solstice
%% %% , that takes advantage of skew and sparsity in typical datacenter traffic
%% %% demand matrices to
%% that provides a 2.9$\times$ increase in circuit utilization over traditional
%% scheduling algorithms, while being within 14\% of optimal, at scale.
