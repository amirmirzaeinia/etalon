\begin{figure*}[t]
  \centering
  \begin{tabular}{| l | c | c | c | c | c |}
    \toprule
    & Multicast & Network Reordering & Anycast & App Reordering & ADUs\\
    \midrule
    MapReduce & & & & &\\
    DFS & & & & &\\
    VM Migration & & & & &\\
    Search & & & & &\\
    Combined & & & & &\\
    \bottomrule
  \end{tabular}
  \caption{Comparing workloads based on which optimizations will improve them.}
  \label{tab:optimizations-to-workloads}
\end{figure*}

\newpage\phantom{t}
\section{Optimizations}
\label{sec:optimizations}

\TODO{point to table.}
\TODO{pitch this as a continuum of painfulness -- e.g., multicast requires
  application restructuring and potentially changes to the fundamental
  communication paradigm / algorithm, compared to ADUs which can be added to
  send call boundaries, etc.} \srini{
  need to have some general discussion of transparency vs. performance:
  e.g. network level optimizations are transparent to higher layers. Also, while
  ADUs pass information down - what is done with them is transparent. Modifying
  application algorithms is the most intrusive but creates the greatest
  opportunity. It might be worth creating some ``toy'' apps that highlight these
  differences and are easy to simulate. E.g. aggregation - perhaps tree vs. ring
  vs flood might provide some insights into their interactions with topology
  limitations and technology capabilities. }

\subsection{Network-level Optimizations}
\label{sec:net_opts}

\TODO{lead in.}

{\bf Multicast:} Multicast allows for redundant copies of data to avoid
transiting the NTS. Of the workloads describe in \S~\ref{sec:workloads}, DFS
most clearly benefits from multicast. If destination nodes reside in different
racks, the NTS must support multicast natively for there to be benefit, however,
if the destination nodes are all in the same rack, having the ToRs support
multicast is sufficient. \TODO{more... Also, is MC a ``network'' optimization or
  more of an ``cross-layer'' optimization?}

{\bf Network Reordering:} Network reordering focuses on moving data around in
hosts buffers to better align with what the network chooses as a schedule for
the next epoch. For example, if the network scheduler connects rack A to rack B,
hosts in rack A should send off all stored data destined for hosts in rack B
regardless of these packets spot in the transmission queue. \TODO{more...}
\TODO{this needs a motivating scenario, maybe a figure. e.g., buffering of 30MB
  per switch port is not that reasonable. 802.3q}

\newpage\phantom{t}
\subsection{Cross-layer Optimizations}
\label{sec:cross_opts}

The cross-layer optimizations we focus on look at the following: 1) who the
application sends data to, 2) when the application sends data, 3) what data the
application sends.

{\bf Anycast:} The anycast optimization focuses on pushing the decision of
destination node to the network. This allows the network scheduler to
potentially avoid sending traffic between racks that are currently disconnected
in the topology. For example, dynamically selecting which nodes to use in a
MapReduce job to make better use of the network is a possibility when using
anycast. \TODO{more...}

{\bf Application Reordering:} Application reordering focuses on changing the
order of the byte stream sent to the network. The network has a very limited
view of future data transmission when compared to applications (especially large
distributed applications like MapReduce). By rewriting applications to send
bursts of traffic to specific destinations (or racks) rather than cycle through
destinations we can greatly simplify network scheduling \TODO{more...}

{\bf ADUs:} Application Data Units (ADUs) are markers placed in the application
data stream to signal where the end of a ``chunk'' of data important to the
application is. This for example could be the end of a query or
transaction. These markers help atomicize each chunk of data transmitted; an
application can proceed only if \emph{all} of the data in this ADU is
transmitted. These markers provide very clear signals to the network
scheduler. \TODO{more...}
\srini{should discuss coflow here}
