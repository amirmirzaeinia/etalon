%-- place any standard commands/environments here to get included in
%-- documents.  When you include this file, you should do it before
%-- the \begin{document} tag.

%%%%%%%%%%%%%%%%%%%%%%%%%%%%%%%%%%%%%%%%%%%%%%%%%%%%%%%%%%%%%%%%%%%%%%
%-- CHANGES:
%-- 07/31/01 -jstrunk- Added command to set the paper margins.

%-- Provides fixed width font for commands and code snips.
\newcommand{\code}[1]{\texttt{\textbf{#1}}}

%-- Terms...  Use this to introduce a term in the paper.
\newcommand{\term}[1]{\emph{#1}}

%-- Provides stylization for e-mail addresses
%\newcommand{\email}[1]{\emph{(#1)}}

%-- Starts a minor section (puts the title inline w/ the text.
\newcommand{\minorsection}[1]{\textbf{#1}:}

%-- Jiri caption
\newcommand{\minicaption}[2]{\caption[#1]{\textbf{#1.} #2}}

%-- Units on numbers: 4KB -> \units{4}{KB}
\newcommand{\units}[2]{#1~#2}

%-- Commands...  i.e. WRITE commands.
\newcommand{\command}[1]{{\sc \MakeLowercase{#1}}}

%-- For notes about things that need to be fixed.
\newcommand{\fix}[1]{\marginpar{\LARGE\ensuremath{\bullet}}
    \MakeUppercase{\textbf{[#1]}}}
%-- For adding inline notes to a draft preceded by your initials
%-- E.g., \fixnote{JJW}{What the heck is a foobar?}
\newcommand{\fixnote}[2]{\marginpar{\LARGE\ensuremath{\bullet}}
    {\textbf{[#1:} \textit{#2\,}\textbf{]}}}

%-- Setting margins: \setmargins{left}{right}{top}{bottom}
\newcommand{\setmargins}[4]{
    % Calculations of top & bottom margins
    \setlength\topmargin{#3}
    \addtolength\topmargin{-.5in}  %-- seems like this should be 1, but .5
                                   %-- balances the text top to bottom
    \addtolength\topmargin{-\headheight}
    \addtolength\topmargin{-\headsep}
    \setlength\textheight{\paperheight}
    \addtolength\textheight{-#3}
    \addtolength\textheight{-#4}

    % Calculations of left & right margins
    \setlength\oddsidemargin{#1}
    \addtolength\oddsidemargin{-1in}
    \setlength\evensidemargin{\oddsidemargin}
    \setlength\textwidth{\paperwidth}
    \addtolength\textwidth{-#1}
    \addtolength\textwidth{-#2}
}

%-- For the tabularx environment... Using L, C, R as the column type
%-- will left, center, or right justify the text.
\newcolumntype{L}{X}
\newcolumntype{C}{>{\centering\arraybackslash}X}
\newcolumntype{R}{>{\raggedleft\arraybackslash}X}

%-- To comment out a swatch of text, use \omitit{blah blah blah}
\long\def\omitit#1{}

%-- Inline title; useful for sub-sub-sections in which you don't want a separate
%-- line for the title.
\newcommand{\inlinesection}[1]{\smallskip\noindent{\textbf{#1.}}}
